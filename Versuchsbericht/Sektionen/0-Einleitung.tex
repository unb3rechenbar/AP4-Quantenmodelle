\documentclass[../main.tex]{subfiles}
\begin{document}

%\begin{figure}[H]
%    \centering
%    \includegraphics[width=10cm]{Bilddateien/CoffinDance.jpg}
%    \label{fig:myfreshbild}
%\end{figure}

Die klassische Mechanik gründet sich auf der Newtonschen Bewegungsgleichung $F = \dot p$. einen analogen Stellwert hat die Schrödingergleichung $\hat H(\Psi) = \hat E(\Psi)$ in der Quantenmechanik. Um sich mit der mathematische Struktur dieser Gleichung vertrauter zu machen, wurde in diesem Bericht eine mathematisch ähnliche Differentialgleichung betrachtet: die Helmholtzgleichung, welche Druckwellen beschreibt.\\

Untersucht werden konkret stabile Druckwellen im Inneren eines Metallrohrs, um einfache Vorhersagen der Helmholtzgleichung zu bestätigen. Anschließend werden stabile Schallmoden in einer hohlen Metallkugel untersucht auf die Druckamplitude auf der Kugeloberfläche sowie auf die Resonanzfrequenz. Denn diese Moden sind eng verwandet mit den stabilen quantenmechanischen Zuständen des Wasserstoffatoms.\\

Zuletzt werden Rohr- und Kugelresonatoren betrachtet, die durch Blendenöffnungen miteinander gekoppelt sind. Dadurch kann ein Wasserstoffmolekül, sowie das Verhalten von Molekülorbitalen simuliert werden. Der Rohresonator dient zudem der Darstellung eines gitterperiodischen Festkörpers, welcher ein ähnliches Potential besitzt.

\end{document}