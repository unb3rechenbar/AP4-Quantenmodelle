\documentclass[../main.tex]{subfiles}
\begin{document}
    \subsection{Aspekte der klassischen (Wellen-)Mechanik}
        \subsubsection{Schallwellen}
            % -> stehende Wellen
            % -> Resonanz
            Als Schallwelle bezeichnen wir die lineaere Ausbreitung einer Stoerung in einem Medium. Dies bedeutet die periodische Abstandsänderung der Teilchen in diesem. 

        \subsubsection{Harmonischer Oszillator}
            % -> Amplitudenfunktion
            Der \emph{harmonische Oszillator} ist ein Modell einer Schwingung mit Rückstellkraft. Die Bewegungsgleichung lautet
            \begin{equation}
                \ddot{x} + \omega^2 x = 0
            \end{equation}



    \subsection{Aspekte der Quantenmechanik}
        \subsubsection{Das freie Teilchen}
            % -> Schrödingergleichung


        \subsubsection{Das Potentialtopf-Modell}
            % -> Erweiterung der Schrödingergleichung um ein Potential

        \subsubsection*{Die Dispersionsrelation}


        \subsubsection{Der kristalline Festkörper}
            % -> Allgemeiner Aufbau
            % -> Bandstruktur


        \subsubsection{Das Wasserstoffatom}
            % -> Schrödingergleichung
            % -> Lösungen (Kugelflächenfunktionen)
            % -> Spektrum
            % -> Entartung


        \subsubsection*{Wasserstoffmoleküle}
            % -> Molekülorbitale
            % -> Energieniveaus

    
\end{document}