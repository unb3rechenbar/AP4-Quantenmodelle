\documentclass[../main.tex]{subfiles}
\begin{document}

\paragraph{Der Rohrresonator}
    Durch die Betrachtung von stabilen Druckverteilungen in einem Rohrresonator wurde die Schallgeschwindigkeit bei Zimmertemperatur als $c=\SI{346.28(28)}{\metre\per\second}$ bestimmt, was in $\SI{100}{\%}$ Übereinstimmumg mit dem Literaturwert von $\SI{344}{\metre\per\second}$ bei $\SI{20}{\degree}$ ist. Dies bestätigt zweierlei:
    \begin{itemize}
        \item die Quantisierung der Wellenvektoren bzw. der Energien von Druckmoden im Rohrresonator - was mathematisch analog ist zu der Quantisierung der Energien von gebundenden Zuständen im Potentialtopf mit unendlich hohen Potentialwänden.
        \item die Schallgeschwindigkeit von Druckwellen in Luft. 
    \end{itemize}

    Desweiteren wurde bei der Lebensdauer $\tau$ der stabilen Druckmoden ein chaotischer Zusammenhang mit einer insgesamt negativen Korrelation zwischen Frequenz und $\tau$ festgestellt. Ersteres liegt vermutlich an der gerinden Datenmenge, letzteres steht aber in Übereinstimmumg, mit folgenden Theorien:
    \begin{itemize}
        \item Angeregte quantenmechanische Zustände, also solche, die nicht im Grundzustand sind, haben eine endliche Lebensdauer wegen spontaner Zerfälle
        \item je größer die Frequenz, also Energie, eines Resonanzzustands ist, desto instabiler ist dieser und zerfällt schneller.
    \end{itemize}


\paragraph{Der Kugelresonator ohne Zwischenringe}
    Eine Analyse der Druckmoden in einen einfachen Kugelresonator ergab, dass die Winkelabhängigkeiten der Druckamplituden an der Oberfläche des Resonators qualitativ den Druckverteilungen der Orbitale des Wasserstoffatoms gleichen. Da sich das Potential für Druck in einem Medium und für Elektronen im Wasserstoffatom im Winkelanteil gleichen, aber im Radialteil unterscheiden, wurde somit bestärkt: die Laplace-Gleichung hat tatsächlich Kugelflächenfunktionen als physikalische Lösungen im Winkelanteil.\\

    Ebenso zeigten die Druckverteilungen aber auch keine exakten Knoten im Gegensatz zu Kugelflächenfunktionen; dies liegt wohl an der Entartung in der Quantenzahl $m$, wodurch sich Zustände mit anderem $m$ aber gleichen $l$ überlagern. Die Aufspaltung der Zustände verschiedener $l$-Quantenzahlen verdeutlicht auch den Unterschied zwischen der Helmholtzgleichung und der Schrödingergleichung für das Wasserstoffatom. Der verschiedene Radialteil bei beiden Gleichung führt nämlich zu einer Entartung in der Quantenzahl $l$ bei der Schrödingergleichung, wohingegen diese Entartung bei der Helmholtzgleichung nicht vorliegt.

\paragraph{Der Kugelresonator mit Zwischenringen}
    Wird ein Zwischenring inmitten des Kugelresonators gelegt, so wird die Symmetrie des Systems deutlich stärker als zuvor, durch Lautsprecher und Mikrofon, gebrochen. Diagramm \ref{fig:II_f_Spektren} verdeutlicht dies durch die Aufhebung der Entartung stabiler Zustände in der Quantenzahl $m$ (vergleiche den Zeeman-Effekt beim Wasserstoffatom). Durch die Anzahl der Peaks, in der ein einzelner Peak wegen der Symmetriebrechung aufgespalten wird, wurde erfolgeich die Quantenzahl $l$ bestimmt. Dies wird unterstützt durch verschiedene Azimutalschnitte in \ref{fig:III_jk_Azimutalschnitte}.\\

    Weiter wurde quantitativ in \ref{fig:III_n_Frequenzaufspaltung} eine lineare Abhängigkeit zwischen Ringdicke und Aufspaltung der Resonantfrequenzen festgestellt.

\paragraph{Das Wasserstoffmolekül}
    Eine Kopplung von zwei Kugelresonatoren zeigte zunächst, dass ein stärkerer Kopplungsgrad (also ein größerer Durchmesser der Blende zwischen oberen und unteren Kugelresonator) eine positive Verschiebung von Resonanzenfrequenzen hervorruft. Dies stimmt sowohl mit der Theorie von Druckverteilungen also auch mit der Kopplung quantenmechanischer Zustände überein. Auch die Phasenveschiebung des Zustands bei ca. $f=\SI{450}{\hertz}$ (\ref{fig:IV_a_Spektren}) von $\SI{0}{\degree}$ identifizert diesen Zustand als bindend, was er nach bekannten Energieschemas sein sollte. Bei Resonanzen höherer Energie (\ref{fig:IV_j_Phasendiagramme_2290}, \ref{fig:IV_j_Phasendiagramme_2306}, \ref{fig:IV_j_Phasendiagramme_2456}) zeigten sich jedoch Diskrepanzen zwischen theoretischen Vorhersagungen und Messdaten.

\paragraph{Der Rohrresonator mit Blenden}
    Zuletzt wurden Dispersionsrelationen von stabilen Druckwellen im Rohrresonator mit und ohne Kopplung bestimmt. Hier zeigte sich zum einen der erwartete lineare Verlauf ohne Kopplung (siehe \ref{fig:V_j_Dispersionsrelation_Lang}, \ref{fig:V_l_Dispersionsrelation_Kurz}), zum anderen wurde das Bändermodell mit Kopplung bestätigt. Es zeigten sich regelmäßige Lücken der erlaubten Frequenzen, weiter hatte die Frequenzkurve aufgetragen über den Wellenvektor eine zunehmende $S$-Form.\\

    Auch die Resonansspektren selbst zeigten bei $n$ Blenden zwischen Rohrsegmenten eine Aufspaltung eines Peaks in je $n+1$.

\end{document}