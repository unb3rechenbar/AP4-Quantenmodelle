\documentclass[../main.tex]{subfiles}
\begin{document}
	\begin{thebibliography}{99}
		% ---- Universitätsinhalte ----
		\bibitem{skript} Bernd-Uwe Runge (2020) \emph{Physikalisches Anfängerpraktikum der Universität Konstanz}, Universität Konstanz. Kompiliert am 13. Mai 2022 at 7:11. 
		% \bibitem{Unsicherheiten} Philipp Möhrke, Bernd-Uwe Runge  (2020) \emph{Arbeiten mit Messdaten Eine praktische Kurzeinführung nach GUM}, Springer Spektrum  Berlin. ISBN 978-3-662-60659-9.

		% ---- Junk ----
		% \bibitem{junk:Ana1} Michael Junk (2022). \emph{Analysis 1}. Universität Konstanz. 
		% \bibitem{junk:Ana2} Michael Junk (2022). \emph{Analysis 2}. Universität Konstanz.
		\bibitem{junk:Ana3} Michael Junk (04.02.2022). \emph{Analysis 3 Gewöhnliche Differentialgleichungen}. Universität Konstanz.
		% \bibitem{junk:Maß} Michael Junk (2022). \emph{Maß- und Integrationstheorie}. Universität Konstanz.

		% ==== Bücher ====
		% ---- Nolting ----
		% \bibitem{nolting3} Nolting, W. (2011). \emph{Grundkurs Theoretische Physik 3: Elektrodynamik}. 9. Auflage. Springer Spektrum. ISBN 978-3-642-13448-7.
		% \bibitem{nolting42} Nolting, W. (2016). \emph{Grundkurs Theoretische Physik 4/2: Thermodynamik}. 9. Auflage. Springer Spektrum. ISBN 978-3-662-49033-4.

		% ---- Demtröder ----
		% \bibitem{demtroeder1.9} Wolfgang Demtröder (2021) \emph{Experimentalphysik - Mechanik und Wärme}, 9. Auflage, Springer-Verlag. ISBN 978-3-662-62727-3.
		% \bibitem{demtroeder2.7} Wolfgang Demtröder (2017) \emph{Experimentalphysik 2 - Elektrizität und Optik}, 7. Auflage, Springer-Verlag. ISBN 978-3-540-79294-9.
		
		% ---- Heintze ----
		% \bibitem{heintze} Joachim Heintze. (2016). \emph{Lehrbuch zur Experimentalphysik 2: Kontinuumsmechanik und Thermodynamik}. Springer Spektrum. ISBN 978-3-662-45767-2.

		% ---- Metzler ----
		% \bibitem{Metzler2} Grehn, Joachim und von Hessberg, Albrecht und Holz, Hans-Gerd und Krause, Joachim und Krüger, Herwig und Schmidt, Hans Kurt (1979). \emph{Metzler Physik}. 2. Auflage, Druck von 1990. J.B.Metzlersche Verlagsbuchhandlung und Carl Ernst Poeschel Verlag GmbH, Stuttgart. ISBN 3-476-50209-0.
		% \bibitem{Metzler3} Bolz, Dr. Joachim und Grehn, Joachim und Krause, Joachim und Krüger, Herwig und Kurt Schmidt, Dr. Herbert und Schwarze, Dr. Heine (1998). \emph{Metzler Physik}. 3. Auflage. Schroedel Verlag GmbH, Hannover. ISBN 3-507-10700-7.

		% ---- Haken Wolf ----
		% \bibitem{HakenWolf} H. Haken, H. C. Wolf (2004). \emph{Atom- und Quantenphysik}. 8. Auflage. Springer-Verlag. ISBN 978-3-642-62142-0.

		% ---- Schneider ----
		% \bibitem{Schneider} Wener B. Schneider (1991). \emph{Wege in der Physikdidaktik - Band 2}. Verlag Plam \& Enke. ISBN 3-7896-0100-4.

		% ---- Claude Cohen-Tannoudji ----
		% \bibitem{QuantumMechanics} Claude Cohen-Tannoudji, Bernard Diu, Franck Laloë (2020). \emph{Quantum Mechanics}. 2. Auflage. Wiley-VCH. ISBN 978-3-527-34553-3.

		% ==== Webseiten ====
		% ---- Nist ----
		% \bibitem{nist:gasconstant} National Institute of Standards and Technology (2022). \emph{Molar gas constant}. URL \url{https://physics.nist.gov/cgi-bin/cuu/Value?r|search_for=gas+constant}, last visited on 2022-05-13.

		% ---- Wikipedia ----
		\bibitem{wiki:Quantenzahl} Wikipedia (2023). \emph{Quantenzahl --- Wikipedia{,} Die freie Enzyklopädie}. URL \url{https://de.wikipedia.org/w/index.php?title=Quantenzahl&oldid=232044429}, zuletzt abgerufen am 06.08.2023. 
		\bibitem{wiki:Laguerrepolynome} Wikipedia (2022). \emph{Laguerre-Polynome --- Wikipedia{,} Die freie Enzyklopädie}. URL \url{https://de.wikipedia.org/w/index.php?title=Laguerre-Polynome&oldid=220377470}, zuletzt abgerufen am 06.08.2023. 
		\bibitem{wiki:Wasserstoffatom} Wikipedia (2023). \emph{Wasserstoffatom --- Wikipedia{,} Die freie Enzyklopädie}. URL \url{https://de.wikipedia.org/wiki/Wasserstoffatom#Mathematische_Details}, zuletzt abgerufen am 06.08.2023.
		\bibitem{wiki:Kugelflaechenfunktionen} Wikipedia (2023). \emph{Kugelflächenfunktionen --- Wikipedia{,} Die freie Enzyklopädie}. URL \url{https://de.wikipedia.org/w/index.php?title=Kugelflächenfunktionen&oldid=235403214}, zuletzt abgerufen am 06.08.2023.
		\bibitem{wiki:zugeordneteLegendrepolynome} Wikipedia (2023). \emph{Zugeordnete Legendrepolynome --- Wikipedia{,} Die freie Enzyklopädie}. URL \url{https://de.wikipedia.org/w/index.php?title=Zugeordnete_Legendrepolynome&oldid=235573320}, zuletzt abgerufen am 06.08.2023.
		\bibitem{wiki:Legendrepolynome} Wikipedia (2023). \emph{Legendre-Polynom --- Wikipedia{,} Die freie Enzyklopädie}. URL \url{https://de.wikipedia.org/w/index.php?title=Legendre-Polynom&oldid=236008036}, zuletzt abgerufen am 06.08.2023.
		\bibitem{wiki:BohrRadius} Wikipedia (2023). \emph{Bohrscher Radius --- Wikipedia{,} Die freie Enzyklopädie}. URL \url{https://de.wikipedia.org/w/index.php?title=Bohrscher_Radius&oldid=235791764}, zuletzt abgerufen am 06.08.2023.

		% ---- GitHub ----
		\bibitem{git:IK4T} GitHub (2023). \emph{IK4-TheSkript}. URL \url{https://github.com/unb3rechenbar/IK4-TheSkript}, zuletzt abgerufen am 09.06.2023.
		% \bibitem{git:IK4E} GitHub (2023). \emph{IK4-ExpSkript}. URL \url{https://github.com/unb3rechenbar/IK4-ExpSkript}, zuletzt abgerufen am 09.06.2023.
	

	\end{thebibliography}
\end{document}